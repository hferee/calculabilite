
\section{Introduction}


Dans ce cours, on s'intéresse à la notion de calculabilité, mais pour cela, il faut comprendre ce qu'est un problème.
D'une manière informelle, on peut voir un problème comme une fonction qui prend en entrée des données et retourne une réponse binaire.
On fait la distinction entre un problème et une fonction qui, au lieu de retourner une réponse binaire, renvoie un nouvel ensemble de données.
\begin{definition}
	Un \textbf{problème} est une fonction $f: \Sigma^* \to \{0, 1\}$, où $\Sigma$ est un alphabet fini.
\end{definition}

\begin{remarque}
	L'ensemble des problèmes est infini dénombrable et il est en bijection avec $\parts \Sigma^*$.
\end{remarque}

\begin{remarque}
	L'ensemble des fonctions (en utilisant la définition de fonction donnée) est l'ensemble des fonctions $f: \Sigma^* \to \Sigma^*$ qui est infini non dénombrable,
	et donc bien plus grand que l'ensemble des problèmes. De plus, l'ensemble des problèmes est un sous-ensemble de l'ensemble des fonctions à isomorphisme près.
\end{remarque}

Il nous manque maintenant la notion de programme pour pouvoir parler de calculabilité.

En 1900, David Hilbert a posé 23 problèmes mathématiques qu'il considérait comme les plus importants de son époque. L'un de ces problèmes, le 10e, consistait à trouver
une méthode (un nombre fini d'étapes) pour décider si une équation diophantienne a une solution entière. Cependant, aucune méthode n'a été trouvée, car il n'en existe pas.
Pouvoir faire ce genre de démonstration relève de la calculabilité.

Regardons quelques exemples de constructions qui sont normalement associées à la notion de programme :

\begin{itemize}
	\item Les langages de programmation comme Python, Java, C, etc. \circled 3
	\item Le $\lambda$-calcul \circled 3
	\item Les machines de Turing \circled 3
	\item Les automates $\iff$ les expressions régulières \circled 1
	\item Les automates à pile $\iff$ les grammaires hors-contexte \circled 2
	\item Les automates à 2 ou plusieurs piles \circled 3
	\item Les transducteurs
\end{itemize}

Ces constructions sont classées selon la hiérarchie de Chomsky \ref{fig:chomsky}.

\begin{figure}[!htb]
	\centering
	% From: https://tex.stackexchange.com/questions/484541/nested-ellipses-in-tikzpicture-chomsky-hierarchy
	\begin{tikzpicture}[font=\sffamily,breathe dist/.initial=2ex]
		\foreach \X [count=\Y,remember=\Y as \LastY] in
			{Régulières \circled 1, Hors contexte \circled 2, Context sensitive, Récursivement énumerables \circled 3}
			{\ifnum\Y=1
					\node[ellipse,draw,outer sep=0pt] (F-\Y) {\X};
				\else
					\node[anchor=south] (T-\Y) at (F-\LastY.north) {\X};
					\path let \p1=($([yshift=\pgfkeysvalueof{/tikz/breathe dist}]T-\Y.north)-(F-\LastY.south)$),
					\p2=($(F-1.east)-(F-1.west)$),\p3=($(F-1.north)-(F-1.south)$)
					in ($([yshift=\pgfkeysvalueof{/tikz/breathe dist}]T-\Y.north)!0.5!(F-\LastY.south)$)
					node[minimum height=\y1,minimum width={\y1*\x2/\y3},
							draw,ellipse,inner sep=0pt] (F-\Y){};
				\fi}
	\end{tikzpicture}
	\caption{Hiérarchie de Chomsky}
	\label{fig:chomsky}
\end{figure}


La thèse de Church-Turing est une hypothèse qui postule que la seule notion de problème décidables est \circled 3.


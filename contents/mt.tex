
\section{Machines de Turing}

\subsection{Définition}

\begin{definition}[Machine de Turing]
	Étant donné un alphabet $\Sigma$, une \textbf{machine de Turing} est un 6-uplet $M = (Q, \Gamma, \delta, q_0, q_a, q_r)$ où :
	\begin{itemize}
		\item $Q$ est un ensemble fini d'états
		\item $\Gamma$ est un alphabet fini de symboles de ruban, et $\Sigma \subseteq \Gamma$
		\item $\delta$ est la fonction de transition
		      $$ \delta: \underbrace{Q}_{\text{État currant}} \times \underbrace{\Gamma}_{\text{Lit}} \to \underbrace{Q}_{\text{Nouveau état}} \times
			      \underbrace{\Gamma}_{\text{Écrit}} \times \underbrace{\{R, L, N\}}_{\text{Direction}} $$
		\item $q_0 \in Q$ est l'état initial
		\item $q_a \in Q$ est l'état d'acceptation
		\item $q_r \in Q$ est l'état de rejet
	\end{itemize}
\end{definition}

\begin{definition}[Configuration]
	Une \textbf{configuration} d'une machine de Turing est un triplet $(q, c, pos)$ où :
	\begin{itemize}
		\item $q \in Q$ est l'état courant
		\item $c$ est le contenu du ruban
		\item $pos$ est la position de la tête de lecture
	\end{itemize}
\end{definition}

Comment se déroule une exécution d'une machine de Turing pour un mot $w \in \Sigma^*$ ?

\begin{enumerate}
	\item On initialise le ruban
	      \begin{figure}[!htb]
		      \centering
		      % Inspired from https://tex.stackexchange.com/questions/49839/turing-machine-figure
		      \begin{tikzpicture}[every node/.style={block},
				      block/.style={minimum height=1.5em,outer sep=0pt,draw,rectangle,node distance=0pt}]
			      \node (A) {$w_0$};
			      \node (B) [left=of A] {$B$};
			      \node (C) [left=of B] {$\ldots$};
			      \node (D) [right=of A] {$\ldots$};
			      \node (E) [right=of D] {$w_n$};
			      \node (F) [right=of E] {$B$};
			      \node (G) [right=of F] {$\ldots$};
			      \node (T) [below = 0.75cm of A] {$q_i$};
			      \draw[-latex] (T) -- (A);
			      \draw (C.north west) -- ++(-1cm,0) (C.south west) -- ++ (-1cm,0)
			      (G.north east) -- ++(1cm,0) (G.south east) -- ++ (1cm,0);
		      \end{tikzpicture}
	      \end{figure}

	\item Si on est dans l'état $q$, que la lettre sous la tête de lecture est $a$ et que $\delta(q, a) = (q', b, d)$, alors on fait :
	      \begin{itemize}
		      \item On écrit $b$ à la place de $a$
		      \item On se déplace dans l'état $q'$
		      \item On déplace la tête de lecture dans la direction $d$
	      \end{itemize}
	\item Si on arrive dans l'état $q_a$ ou $q_r$, alors on arrête l'exécution. Si on arrive dans l'état $q_a$, alors on accepte le mot $w$ et si on arrive dans l'état $q_r$, alors on rejette le mot $w$.
\end{enumerate}


\begin{notation}
	Soit $M$ une machine de Turing, $w \in \Sigma^*$ un mot, alors on note $M(w)$ l'exécution de $M$ sur $w$. Cette exécution peut être :
	\begin{itemize}
		\item Acceptée : $M(w) = 1$
		\item Rejetée : $M(w) = 0$
		\item Bouclée : $M(w) = \bot$
	\end{itemize}
\end{notation}


\begin{remarque}
	Les automates s'injectent dans les machines de Turingm il suffit d'ignorer le ruban et de considérer que la tête de lecture est fixe.
\end{remarque}

\begin{definition}[Machine de Turing non déterministe]
	De manière analogue aux automates la notion de \textbf{machine de Turing non déterministe} étend la définition d'une machine de Turing en
	permettent d'avoir plusieurs transitions pour un état donné. La différence se trouve donc dans le type de la fonction de transition :

	$$ \Delta: Q \times \Gamma \to \parts{Q \times \Gamma \times \{R, L, N\}}$$

\end{definition}


\subsection{Notion de calculabilité}


\begin{definition}[Langage semi-décidable]
	Un langage $L \subseteq \mots$ est \textbf{semi-décidable} s'il existe une machine de Turing $M$ \tlq
	$$ \forall w \in \mots, w \in L \iff M(w) = 1 $$
\end{definition}

\begin{definition}[Langage décidable]
	Un langage $L \subseteq \mots$ est \textbf{décidable} s'il existe une machine de Turing $M$ \tlq
	$$ \forall w \in \mots, w \in L \implies M(w) = 1 \quad \text{et} \quad w \notin L \implies M(w) = 0 $$
	et $M$ s'arrête pour tout $w$.
\end{definition}

\begin{prop}
	Tout langage décidable est semi-décidable.
\end{prop}

\begin{proof}
	Il suffit de monter que si $M$ est une machine de Turing qui décide $L$, alors $M(w) = 1 \iff w \in L$.
	\begin{itemize}
		\item $w \in L \implies M(w) = 1$ est vrai par définition.
		\item $M(w) = 1 \implies w \in L$ peut être montrée par contraposée. Si $w \notin L$, alors $M(w) = 0$ car $M$ décide $L$ et donc $M(w) \neq 1$.
	\end{itemize}
\end{proof}

\begin{definition}[Fonction calculable]
	$f : \mots \to \mots$ est calculable si $\exists M$ \tq $\forall w \in \mots$ $M$ s'arrête sur $w$ avec $f(w)$ sur le ruban.
\end{definition}

\begin{lemme}
	La fonction $succ : \mots \to \mots$ est calculable.
\end{lemme}

\begin{prop}
	$L$ est décidable $\iff$ sa fonction caractéristique est calculable.
\end{prop}

\begin{definition}
	$f : \mots \to \mots$ énumère $L \subseteq \mots$ si $\im f = L$ \ie $\forall w \in L \iff \exists w' \in \mots, f (w') = w$.
\end{definition}

\begin{prop}
	$L$ est récursivement énumerable $\iff$ $L$ est décidable.
\end{prop}

On peut encoder toute machine de Turin par un nombre, qui est appété le nombre de Gödel et noté : $\encode M \in \mots$.

\begin{definition}[Énumération des fonctions calculables]
	$\forall n \in \N, \ \phi_n$ est la fonction calculée par la $n$-ième machine de Turing.
	$$\phi_{\encode M} (w) = M (w)$$
\end{definition}

\subsection{Quelques problèmes}

Quelques problèmes décidables en rapport aux automates :

\begin{itemize}
	\item $\text{ACCEPT}_A = \setdef {<A,w>} {A \text {est un automate qui accepte } w}$.
	      Dire que c'est problème est décidable revient a dire que $\exists$ un interpréteur d'automates en Machine de Turing.
	\item $\text{EQUIV}_A = \setdef {<A,A'>} {A \text { et } A' \text{ acceptent le meme langage}}$.
	\item $\text{EXISTS}_A = $ Il existe un mot reconnu.
	\item $\text{INFINITE}_A = $ l'automate est infini.
\end{itemize}

Cependant, ces problèmes étendus aux automates a piles, ne restent pas tous décidable.
$\text{ACCEPT}_{A_p}$ et $\text{INFINITE}_{A_p}$ restent décidables mais pas $\text{EQUIV}_{A_p}$.

\begin{definition}[Problème de l'arrêt]
	Le problème de l'arrêt est définit comme suit $HALT = \setdef {\encode{M,w}} {M \text{ s'arrête sur } w}$.
\end{definition}

\begin{lemme}[Machines universelles]
	Il existe une machine universelle $U$, \ie, $$\forall M,w, \  U (\encode {M,w}) = M (w)$$
	ou de manière equivalente
	$$\exists u, \forall n,w, \ \phi_n(w) = \phi_u(\encode {n,w})$$
	lemme\end{lemme}

\begin{proof}
	Admisse.
\end{proof}

\begin{prop}
	$HALT$ est semi-décidable.
\end{prop}

\begin{proof}
	Il suffit d'écrire un programme "impératif" :

	Entrée : $\encode {M,w}$

	Code:
	u(M,w);
	return 1;

\end{proof}

\begin{prop}
	HALT est indécidable.
\end{prop}

\begin{proof}
	Supposons par l'absurde que HALT est décidable. Alors il existe un entier $n$ \tq $\phi_n$ décide HALT, c'est-à-dire
	$ \phi_n(\encode {M, w} ) = 1 \iff \phi_n(w) \neq \bot $ et vaut $0$ sinon.

	Soit $n'$ le code de la fonction qui sur $w$ vaut 1 si $\phi_n (w,w) = 0 $ et n'est pas définie sinon. Alors
	\begin{eqnarray*}
		\phi_{n'}(n') = 1 & \text{si} & \phi_n(n',n') = 0\\
		&\text{\ie}& \phi_{n'}(n') \text{ ne s'arrête pas} \\
		&\text{\ie}& \phi_{n'}(n') = \bot
	\end{eqnarray*}


	\begin{eqnarray*}
		\phi_{n'}(n') = \bot & \text{si} & \phi_n(n',n') = 1\\
		&\text{\ie}& \phi_{n'}(n') \neq \bot \quad \lightning
	\end{eqnarray*}

\end{proof}

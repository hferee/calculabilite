\section{Hiérarchie arithmétique}

\subsection{Les ensembles $\Sigma_n, \Pi_n \et \Delta_n$}

Dans cette section, nous allons étudier les $\underbrace{\text{formules}}_{\land, \lor, \lnot}$
$\underbrace{\text{arithmétiques}}_{\N, +, *, \leq, \text{S}}$ du $\underbrace{\text{premier ordre}}_{\exists x, \forall x}$
sous forme prénexe et voir comment les classifier.


\begin{definition}[Forme prénexe]
	Une formule est sous forme prénexe si tous ses quantificateurs non-bornés sont en tête (i.e. "à gauche").

	$(\forall x (x \leq 2)) \land  (\forall y (2 \leq y))$ n'est pas sous forme prénexe mais $(\forall x \forall y ((x \leq 2) \land  (2 \leq y))$ l'est.
\end{definition}

\begin{definition}
	On définit les ensembles :

	\begin{eqnarray*}
		\Sigma_{n + 1} &=& \setdef { \exists x \psi } {\psi \in \Pi_n}  \\
		\Pi_{n + 1} &=& \setdef { \forall x \psi } {\psi \in \Sigma_n}  \\
		\Delta_{n + 1} &=& \Sigma_{n + 1} \cap \Pi_{n + 1} \\
		\Delta_0 &=& \Pi_0 = \Sigma_0 \reason {formules sans quantificateurs non-bornés}
	\end{eqnarray*}

	Autrement dit, $\Sigma_n$ est l'ensemble des formules avec $n$ quantificateurs alternés, qui commencent par un quantificateur existentiel.
\end{definition}

\begin{remarque}
	Dans la définition précédente, les formules de la forme $\forall x \forall y \ldots$ ne sont pas prises en compte. Ceci est dû au fait que, grâce
	au lemme suivant, les quantificateurs égaux qui se suivent peuvent être regroupés en un seul.
\end{remarque}

\begin{lemma}(Admis)
	Pour tout $\phi$ il existe $\phi_1$ et $\phi_2$ tels que :
	$$ \forall x_1, \forall x_2, \phi (x_1,x_2) \iff \forall x, \phi(\phi_1(x), \phi_2(x))$$
	Le même résultat peut être obtenu pour le quantificateur existentiel.
	Cette propriété découle du fait que $\N \cong \N \times \N$ et qu'il existe un tel encodage des paires d'entiers qui peut être décodé 
	via une formule $\Delta_0$. On pourra trouver quelques exemples de fonctions de couplage comme celle de Cantor ici~\url{https://en.wikipedia.org/wiki/Pairing\_function}).
\end{lemma}

\begin{prop}\label{prop:D0-dec}
	Les langages de $\Delta_0$ sont décidables.
\end{prop}

\begin{proof}
	Soit $\phi \in \Delta_0$, alors, comme elle ne contient pas de quantificateurs non-bornés, on peut essayer toutes les valeurs possibles, car il y en a un nombre
	fini.
\end{proof}


Dans cette section, nous allons étudier les relations entres ces ensembles afin de pouvoir les classifier, aboutissant à la hiérarchie de la Figure \ref{fig:arith-hier}.


\begin{figure}[h]
	\begin{center}
		\includegraphics[height=7cm]{./images/Arithmetic_hierarchy.png}
	\end{center}
	\caption{Hiérarchie arithmétique}
	\label{fig:arith-hier}
\end{figure}


\begin{exercice}
	Montrer que $ACCEPT = \setdef {\encode {M,w}} {M(w) = 1} \in \Sigma_1$
\end{exercice}

\begin{proof}
	$$\phi (M,w) =  \exists n, \ \underbrace{eval(\encode M, w, n) = 1}_{\in \Delta_0} \in \Sigma_1$$

	Ainsi, $\phi$ décrit $ACCEPT$ et $\phi \in \Sigma_1$ et donc $ACCEPT \in \Sigma_1$.
\end{proof}


\begin{exercice}
	Montrer que $ACCEPT$ est $\Sigma_1$-complet, \ie
	$$ \forall L \in \Sigma_1, L \leqm ACCEPT $$
\end{exercice}

\begin{proof}
	Soit $L \in \Sigma_1$. Alors il existe $\phi \in \Delta_0$ \tq
	$$ L = \setdef {n} {\exists x \phi(x,n)} $$

	Soit $M(n)$ la machine qui énumère tous les $x$ et accepte si $\phi(x,n)$ est satisfaite.
	Alors $M$ reconnait $L$.
\end{proof}

\begin{prop} \label{prop:sigma-re}
	$RE = \Sigma_1$
\end{prop}

\begin{proof}
D'après la proposition~\ref{prop:sd-spec} :
	$$L \text{ est r.e.} \Rightarrow \exists L_d \text{ décidable }, L = \setdef {w} {\overbrace{\exists w', \underbrace{\encode {w, w'} \in L_d}_{\in \Delta_0}}^{\in \Sigma_1}}$$
Inversement, $$L \in \Sigma_1 \iff L = \{w\ |\ \exists w', \phi(w, w')\} \text{ avec } \phi \in  \Delta_0$$.
Semi-décider $w\in L$ revient alors à énumérer les $w'$ jusqu'à trouver en trouver un tel que $\phi (w, w')$, que l'on peut décider d'après la proposition~\ref{prop:D0-dec}.
	Et donc $RE = \Sigma_1$
\end{proof}

\subsection{Machines à oracle}

\begin{definition}[Machine de Turing à oracle]
	Une machine à oracle est une machine de Turing qui a acces à une fonction $\fmots {\mathscr O}$ pendant son exécution. Elle a donc, en plus de la machine
	initiale, un ruban d'appel (pour l'oracle) et un état spécial d'appel.

	Si la machine rentre dans l'état d'appel avec $u \in \mots$ sur le ruban, alors $\mathscr O (u) \in \mots$ est écrit sur le ruban d'appel.
\end{definition}

\begin{definition}[Reconnaissance]
	On dit que $M$ reconnait $L$ relativement à un oracle~$A$ si
	$$ \forall w, M^A (w) = 1 \iff w \in L $$
	où $M^A$ est l'exécution de $M$ avec l'oracle~$A$.
\end{definition}

\begin{definition}[Reduction de Turing]
	On dit que $A \leqt B \iff A$ est décidable relativement à~$B$.
\end{definition}

\begin{remarque}
	On a que $A \leqm B \implies A \leqt B$. En effet, il suffit de construire la machine à oracle qui sur $w$ appelle l'oracle sur $f(w)$, où $f$ est la fonction de reduction.
\end{remarque}

\begin{remarque}
	Pour tout langage $L$, on a que $L \leqt \bar L \et \bar L \leqt L$ et donc $L \equivt \bar L$.
\end{remarque}


\begin{lemma}
	Si $A$ est $C$-complet et $Y \in C$, alors :
	$$X \text{ est } re(Y) \implies X \text{ est } re(A)$$
\end{lemma}

\begin{proof}
	Par définition, $X \text{ est } re(Y)$ s'il existe une machine à oracle $M$ telle que $X$ est reconnu par $M^Y$.
	Comme $A$ est $C$-complet et $Y \in C$, il existe une réduction calculable $f$ de $Y$ vers $A$.
	On pose $N$ la machine à oracle construite à partir de $M$ et qui applique $f$ au ruban d'appel, avan chaque appel à l'oracle.
	Autrement dit, pour tout oracle ${\mathscr O}$ et toute entrée $w$, $N^{{\mathscr O}}(w) = M^{{\mathscr O} \circ f}$.
	On a donc que $N^A$ reconnaît $X$.
	
\end{proof}


\subsection{Hiérarchie arithmétique}

\begin{theorem}[de Post]
	Nous avons les résultats suivants :
	\begin{enumerate}
		\item \label{thm:post-1}
		      \begin{enumerate}
			      \item \label{thm:post-1a}
			            \begin{eqnarray*}
				            L \in \Sigma_{n+1} &\iff& L \text{ est r.e. relativement à un langage }  \Pi_n  \\
				            &\iff& L \text{ est r.e. relativement à un langage }  \Sigma_{n}
			            \end{eqnarray*}

			      \item
			            \begin{eqnarray*}
				            L \in \Pi_{n+1} &\iff& L \text{ est co-r.e. relativement à un langage }  \Sigma_n  \\
				            &\iff& L \text{ est co-r.e. relativement à un langage }  \Pi_n
			            \end{eqnarray*}
		      \end{enumerate}

		\item Il existe un langage $\Sigma_n$-complet, noté $\emptyset^{(n)}$. \label{thm:post-2}
	\end{enumerate}
\end{theorem}


\begin{definition}[eval avec oracle]
	$eval(\encode M, \sigma,w,t)$ évalue $M^{\sigma}(w)$ en un nombre d'étapes inférieur à $t$.
\end{definition}

\begin{proof}[Démonstration du point \ref{thm:post-1a} \bimpRL]
    \ \newline
	$L$ est calculable relativement à $A \in \Pi_n$ via la machine $M_L$.

	\begin{eqnarray*}
		w \in L &\iff& \exists t, M^A \text{ accepte } w \text{ en un temps inférieur à } t\\
		&\iff& \exists t, \exists \sigma,
		\underbrace{\sigma \subseteq A}_{\forall u,v, (u,v) \in \sigma \ra (v=1 \iff \phi_A(u))}
		\land\  M^{\sigma} (w) \text{ accepte en temps inférieur à }t
	\end{eqnarray*}

	Ainsi, si $A \in \Pi_n$ alors $L \in \Sigma_{n+1}$. Sinon il suffit d'écrire
	$v = 0 \iff \underbrace{\lnot \phi_A(u)}_{\in \Pi_n}$ et on a aussi que $L \in \Sigma_{n+1}$.

	Les autres cas du point \ref{thm:post-1} se démontrent de manière similaire.
\end{proof}

\subsubsection{$\Sigma_n$ complétude}

\begin{notation}
	$\phi_{c \in \mots}$ : énumeration des fonctions calculables.

	$\phi_{c \in \mots}^X$ : énumeration des fonctions calculables relativement à $X$. Et donc $\phi_{\encode M}^X(w) = M^X(w)$
\end{notation}

\begin{definition}[Saut de Turing]
	Soit $X$ un langage,

	$X' = \setdef c {\phi_c^X(c) \text{ est défini}} = \setdef {\encode M} {M^X \text{ s'arrête sur } \encode M}$

\end{definition}


\begin{definition}
	$$\emptyset' = \setdef {\encode M} {M^{\emptyset} \text{ s'arrête sur } \encode M}$$
	$$\emptyset^{(n+1)} = \setdef {\encode M} {M^{\emptyset^{(n)}} \text{ s'arrête sur } \encode M} $$
\end{definition}


\begin{remarque}
	$$\emptyset' \equivm \halt$$
\end{remarque}


\begin{proofI}
	\begin{itemize}
		\item \begin{eqnarray*}
			      \encode M \in \emptyset' &\iff& M^{\emptyset} (\encode M) \neq \bot \\
			      &\iff& \encode {M, \encode M} \in \halt
		      \end{eqnarray*}
		      Et donc $\emptyset' \leqm \halt$.

		\item
		      \begin{eqnarray*}
			      \encode {M,w} \in \halt &\iff& M(w) \neq \bot \\
			      &\iff& \encode{\fun {\_} M(w)} \in \emptyset'
		      \end{eqnarray*}
		      Et donc $\halt \leqm \emptyset'$.
	\end{itemize}
\end{proofI}


\begin{exercice}
	Montrer que $\emptyset '$ est $\Sigma_1$-complet :
	\begin{enumerate}
		\item $\emptyset' \in \Sigma_1$
		\item $\forall L \in \Sigma_1, L \leqm \emptyset'$
	\end{enumerate}
\end{exercice}

\begin{proofI}
	\begin{enumerate}
		\item $\encode M \in \emptyset' \iff \exists t, eval (\encode M, ((\_,0), \ldots, (\_,0)), \encode M, t) \neq \bot$
		\item On a que $\emptyset' \equivm \halt$ qui est $\Sigma_1$-complet, donc $\emptyset'$ l'est aussi.
	\end{enumerate}
\end{proofI}




\begin{proof}[Démonstration du point \ref{thm:post-2}]
	Il faut donc montrer que $\forall n, \emptyset^{(n)}$ est $\Sigma_n$-complet.
	La démonstration est faite par induction. Les cas $n=0$ et $n=1$ on été traités précédemment, il suffit alors de montrer le cas d'induction.

	Supposons $\emptyset^{(n)} \ \Sigma_n$-complet.

	\begin{eqnarray*}
		L \in \Sigma_{n+1} &\iff& \exists A \in \Sigma_n, L \text { est r.e. relativement à } A \\
		&\iff& L \text{ est re}(\emptyset^{(n)})\\
		&\iff& \exists M_L, M_L^{\emptyset^{(n)}} \text { reconnait } L \\
		&\iff^? & L \leqm \emptyset^{(n + 1)}
	\end{eqnarray*}

	\begin{itemize}
		\item \bimpLR
		      \begin{eqnarray*}
			      w \in  L  &\iff&  M_L^{\emptyset^{(n)}}(w) \text{ s'arrête et accepte} \\
			      &\iff&  M_{L,w}^{\emptyset^{(n)}}(\cdot) \text{ s'arrête } \\
			      &\iff&  \encode{M_{L,w}} \in \emptyset^{(n+1)} \\
			      &\iff&  L \leqm  \emptyset^{(n+1)}
		      \end{eqnarray*}
		\item \bimpRL \\
		      On a que $L \leqm \emptyset^{(n+1)}$. Soit $f$ la fonction de réduction, alors on pose
		      $M_L = M_{\emptyset^{(n+1)} \circ f}$. Cette machine vérifie bien que $M_L^{\emptyset^{(n)}}$ reconnait  $L$.
	\end{itemize}
\end{proof}


\subsubsection{La hiérarchie ne s'effondre pas}

On s'intéresse maintenant à savoir si la hiérarchie est stricte, \ie, $\exists \phi \in \Sigma_{n+1}, \text{ \tq }, \phi \notin \Sigma_n \et \phi \notin \Pi_n$.


\begin{prop}
	$$\forall n \in \N^*, \emptyset^{(n)} \notin \Delta_n$$
\end{prop}


\begin{remarque}
	Pour $n > 0$
	\begin{eqnarray*}
		L \in \Delta_n &\iff& L \in \Sigma_n \land L \in \Pi_n \\
		&\iff& L \text{ est } re(\Sigma_{n-1}) \land L \text{ est } co-re(\Sigma_{n-1})\\
		&\iff& L \text{ est } re \text{ relativement à } \emptyset^{(n-1)} \land  L \text{ est } co-re \text{ relativement à } \emptyset^{(n-1)} \\
		&\iff& L \text{ est décidable relativement à } \emptyset^{(n-1)}
	\end{eqnarray*}
\end{remarque}

\begin{proof}
	On peut utiliser la remarque précédente. Ainsi, il suffit de montrer que $\emptyset^{(n)}$ n'est pas décidable relativement à $\emptyset^{(n-1)}$.

	Supposons par l'absurde que $\emptyset^{(n)}$ est décidable relativement à $\emptyset^{(n-1)}$, \ie, il existe $M_n$ telle que $M_n^{\emptyset^{(n-1)}}$ décide $\emptyset^{(n)}$.

	Alors on construit

	$$
		M_0(\encode M) =
		\left\{
		\begin{array}{ll}
			\text{Si } M_n^A(\encode M) \text{ accepte} & \rightarrow \text{ boucle}  \\
			\text{Si } M_n^A(\encode M) \text{ rejecte} & \rightarrow \text{ accepte} \\
		\end{array}
		\right.
	$$

	Alors on a que
	\begin{eqnarray*}
		M_0^{\emptyset^{(n-1)}}(\encode{M_0}) \text{ boucle} &\iff& M_n^{\emptyset^{(n-1)}} (\encode {M_0}) \text{ accepte} \\
		&\iff& M_0^{\emptyset^{(n-1)}}(\encode{M_0}) \text{ s'arrête} \ \contradict
	\end{eqnarray*}
\end{proof}


\begin{coro}
	La hiérarchie ne s'effondre pas.
\end{coro}

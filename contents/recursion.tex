\section{Théorèmes de récursion}


\begin{theorem}[d'itération / Smn / d'application partielle]\label{thm:it}
	Il existe une fonction calculable et totale $s$ \tlq
	$$\forall n,m,w, \phi_{s(n,m)}(w) = \phi_m(\encode {n,w})$$

	Si $m$ est le code d'un programme et $n$ un mot, alors $s(n,m)$ est le code de l'application partielle de $\phi_m$ à $n$.
\end{theorem}

\begin{theorem}[de point fixe]
	Si $ \fmots f$ \emph{calculable} et \emph {totale}. Alors il existe $e \in \mots$ \tq $\phi_e = \phi_{f(e)}$.
\end{theorem}


\begin{proof}
	Soit $G$ la machine : $(x,y) \to $ calcule $e = \universal x x$ et retourne $\universal e y$.

	On pose $h(x) = s(\encode G, x)$ (le $s$ du théorème precedent). On a que $f \circ h$ est calculable te totale et notons son coed $c$. Alors

	\begin{eqnarray*}
		\phi_{h(c)} (w) &=& \phi_{s(\encode G, c)} (w) \reason{par définition de $h$} \\
		&=& \phi_{\encode G} (\encode {c, w}) \reason{par \ref{thm:it}} \\
		&=& G (\encode {c, w}) \reason{correspondence énumeration machine }\\
		&=& \letin e {\universal c c} \universal e y \reason{par définition de $G$ }\\
		&=& \phi_{f \circ h (c)}(w) \reason{car $\universal c c = \phi_c(c) = f \circ h (c)$ par définition}
	\end{eqnarray*}
	Donc $\phi_{h(c)} = \phi_{f \circ h (c)}$ et donc $h(c)$ est notre point fixe.
\end{proof}


\begin{theorem}[de récursion]
	Si $f : \mots \times \mots \to \mots$ est une fonction partielle et calculable. Alors il existe une machine $R$ qui calcule $\fmots r $ \tq
	$$ \forall w, r(w) =  f (\encode R, w)$$
\end{theorem}


\begin{proof}
	Soit $M_f$ une machine qui calcule $f$ et $\fmots , \ g(p) = s(\encode {M_f}, p)$
	Alors on applique le théorème de point fixe à $g$ et on a qu'il existe $e$ \tq $\phi_e(w)
		=\phi_{s(\encode{M_f},e)} (w) = \phi_{\encode {M_f}} (e,w) = f(e,w)$
\end{proof}

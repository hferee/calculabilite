\section{Réductions}

\subsection{Réduction many-one}

\begin{definition}[Réduction many-one]
	$A \leqm B$ \ssi $\exists \fmots f$ calculable et totale \tlq
	$$\forall w \in \alphabet, w \in A \iff f(w) \in B$$
\end{definition}


\begin{theorem} \label{thm:leqm_dec}
	$A \leqm B$ et $B$ est décidable, alors $A$ est décidable
\end{theorem}

\begin{proof}

	Soit $M_B$ la machine qui décide $B$ et $M_f$ celle qui calcule $f$. On pose $M_A (w) = M_B(M_f(w))$.

	Montrons que $M_A$ décide $A$. On a les équivalences suivantes:
	\begin{eqnarray*}
		\forall w \in \mots, w \in A &\iff& f(w) \in B \\
		&\iff& M_B (M_f (w)) = 1 \\
		&\iff& M_A \text{ accepte } w
	\end{eqnarray*}

	et de manière analogue

	\begin{eqnarray*}
		\forall w \notin \mots, w \notin A &\iff& f(w) \notin B \\
		&\iff& M_B (M_f (w)) = 0 \\
		&\iff& M_A \text{ rejette } w
	\end{eqnarray*}

	Et donc $M_A$ décide $A$.

\end{proof}


\begin{coro}
	$A \leqm B$ et $A$ n'est pas décidable, alors $B$ n'est l'est pas non plus.
\end{coro}

\begin{proof}
	C'est la contraposée du théorème précédent.
\end{proof}

\begin{theorem}\label{thm:leqm_sdec}
	$A \leqm B$ et $B$ est semi-décidable, alors $A$ est semi-décidable
\end{theorem}

\begin{proof}
	Voir \ref{thm:leqm_dec}
\end{proof}

\begin{theorem}
	$A \leqm B$ et $B$ est co-semi-décidable, alors $A$ est co-semi-décidable.
\end{theorem}

\begin{proof}
	Si $A \leqm B$ alors $\overline A \leqm \overline B$ car
	\begin{eqnarray*}
		(w \notin A &\iff& f(w) \notin B) \\
		&\iff& A \leqm B \reason{par contraposée}
	\end{eqnarray*}
\end{proof}

\begin{definition}
	$A \equivm B \iff A \leqm B \et B \leqm A$
\end{definition}

\subsection{Classes d'équivalence sous many-one}

\begin{definition}[Many-one complet]
	$L$ est many-one complet pour une classe de langages $\mathcal C$ si $\forall L' \in \mathcal C, L' \leqm L$
\end{definition}

\begin{prop}[Problème de Post]
	HALT est RE-complet.
\end{prop}

\begin{proof}
	Soit $L \in $ RE, montrons que $L \leqm \halt$. Soit $M_L$ la machine qui semi-décide $L$.

	Alors on construit $M'_L$ la machine suivante:
	\begin{algorithmic}[lines]
		\Function{$M'_L$}{$w$}
		\If {$M_L (w) = 1$} {$1$}
		\Else { $\bot$}
		\EndIf
		\EndFunction
	\end{algorithmic}

	\begin{eqnarray*}
		w \in L &\iff& M_L = 1 \\
		&\iff& M'_L \neq \bot \\
		&\iff& M'_L \text{ s'arrête sur } w\\
		&\iff& \encode{M'_L,w} \in \halt
	\end{eqnarray*}

	Et donc, on pose $f (w) = \encode {M'_L,w}$ et on a bien que
	$w \in L \iff f(w) \in \halt$ et donc HALT est RE-complet.
\end{proof}

\begin{exercice}
	Construire, à partir de HALT, un langage qui n'est ni RE ni co-RE.
\end{exercice}

\begin{proof}
	Considérons
	$$ L = \setdef {\encode {M,M',w}} {M(w) = \bot \et M'(w) \neq \bot} $$

	Comme HALT et $\overline {\halt}$ sont RE-complet et co-RE-complet respectivement, il suffit de montrer $\halt \leqm L \et \overline{\halt} \leqm L$.

	\begin{itemize}
		\item $\encode {M,w} \in \halt \iff \encode {M_{loop}, M, w} \in L$. Donc $\halt \leqm L$ (où $M_{loop}$ est une machine qui boucle toujours).
		\item $\encode {M,w} \in \overline{\halt} \iff \encode {M, M_1, w} \in L$. Donc $\overline{\halt} \leqm L$ (où $M_1$ est une machine qui s'arrête toujours).
	\end{itemize}

	On a donc que $L$ n'est ni RE ni co-RE.
\end{proof}

Dans les deux exercices suivants, on pourra utiliser (implicitement ou explicitement) le fait que la composition de machines de Turing est \emph{uniformément calculable}.
\begin{lemma}\label{lem:comp}
	Il existe une fonction calculable \emph{comp} telle que :
	$$\forall c c' w, \phi_{\emph{c, c'}}(w) = \phi_c(\phi_{c'}(w))$$
\end{lemma}

% Informellement, il « suffit » de décrire le code de la machine composée de l'union disjointe des deux machines d'entrée
% et de relier l'état final de l'une à l'état initial de l'autre.


\begin{exercice}
	Montrer que l'ensemble des programmes reconnaissant le langage vide est indécidable.
\end{exercice}

\begin{proof}
	Il suffit de montrer que $\overline{\halt} \leqm L_{\emptyset} = \setdef{\encode M} {\forall w \in \mots, M(w) \neq 1}$.

	Notons $f_1$ la fonction calculable qui sur $\encode M$ retourne le code de l'une machine $M'$ définie par:
	\begin{algorithmic}[lines]
		\Function{$M'$}{$w$}
		\State $M(w)$;
		\State \textbf{return} $1$
		\EndFunction
	\end{algorithmic}
	Prour une preuve plus formelle de la calculabilité de cette fonction, on pourra utiliser le lemme~\ref{lem:comp}
	ainsi que le théorème Smn (i.e. théorème~\ref{thm:it}).

	Alors on a que
	\begin{eqnarray*}
		\encode{M,w} \in \overline{\halt} &\iff& M(w) = \bot \\
		&\iff& \forall w, \phi_{f_1(\encode M)}(w) \neq 1 \\
		&\iff& f_1(\encode M) \in L_{\emptyset}
	\end{eqnarray*} et donc $\overline{\halt} \leqm L_{\emptyset}$.

\end{proof}


\begin{exercice}
	Montrer que l'équivalence de machines de Turing n'est ni RE ni co-RE.
\end{exercice}

\begin{proof}
	Il faut montrer que le langage $EQUIV =  \setdef {\encode {M,M'}} {\forall w \in \mots, M(w) = M'(w)}$ vérfie
	$\halt \leqm EQUIV \et \overline{\halt} \leqm EQUIV$.

	Soit $M$ une machine de Turing et $w$ un mot :
	\begin{itemize}
		\item On pose $M' =  \fun \_ {M(w); \text{ return } 1}$. \\
		      $\encode {M,w} \in \halt \iff \encode {M_1, M', w} \in EQUIV$. \\
			La transformation $\encode M \mapsto \encode {M_1, M', w}$ est bien calcualble, donc $\halt \leqm EQUIV$.
		\item  On pose $M' =  \fun \_ {M(w)}$. \\
		      $\encode {M,w} \in \overline{\halt} \iff \encode {M_{loop}, M', w} \in L$. \\
			La transformation $\encode M \mapsto \encode {M_{loop}, M', w}$ est bien calcualble, donc $\overline{\halt} \leqm EQUIV$.
	\end{itemize}

	On a donc, d'après le théorème~\ref{thm:leqm_sdec} que $EQUIV$ n'est ni RE ni co-RE.
\end{proof}


